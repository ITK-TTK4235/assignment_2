





\section{Oppgave (100 \%) - Grunnleggende Git}\label{sec:2-oppgave}

For å få godkjent øvingen, skal dere vise at dere har skjønt det meste av walkthroughen. Dere skal derfor vise hva dere har gjort til en studass og besvare noen spørsmål fra studassen på sal, før dere får øvingen godkjent. Dere oppfordres også til å utforske \verb|git| på egenhånd ved å bruke \verb|git help| eller \verb|git help tutorial|, ettersom dere kommer til å bruke \verb|git| til heisprosjektet (og videre i arbeidslivet).

\section{Oppgave (anbefalt) - GitHub}
Når man bruker \verb|git| i praksis, er det vanlig i kombinasjon med en såkalt \textit{code hosting platform}. GitHub er det mest brukte eksempelet på dette, men det finnes også andre alternativer som Bitbucket, GitLab, mm. I TTK4235 anbefaler vi bruk av GitHub, og som student ved NTNU får man faktisk tilgang til GitHub Pro.

For å ta i bruk GitHub må man først lage seg en bruker, og dette gjøres på nettsiden \href{https://github.com/}{https://github.com/}. Hvis man registrerer seg med stud-mailen gjør dette ting litt enklere når man skal oppgradere til Pro-bruker. Hvordan dette gjøres lar vi dere finne ut av på egenhånd.

Når man har laget seg en bruker og logget inn er det på tide å lage sitt første \textit{repository}. Dette gjøres ved å trykke på den grønne knappen på hovedsiden, hvor det enten står \verb|New| eller \verb|Create repository|. Her får man muligheten til å velge navn under \textit{Repository name} og beskrivelse under \textit{Description}. Man får også muligheten til å bestemme om \textit{repositoryet} skal være \textit{Public} eller \textit{Private}, altså om andre internett-brukere skal kunne se det eller ikke. Når man er fornøyd med valgene sine kan man trykke på \verb|Create repository| for å fullføre prosessen. Deretter kommer man til hovedsiden til ditt nylagde \textit{repository}.

Nå er det på tide å lage et såkalt \textit{token}. Dette er et alternativ til å bruke passord til autentisering, når man bruker kommandolinja, altså terminalen, i Linux. Det finnes to typer \textit{tokens}, nemlig \textit{fine-grained personal access tokens} og \textit{Personal access tokens (classic)
}. Vi skal benytte oss av sistnevnte, ettersom de er bittelitt enklere å bruke. Et \textit{token} knyttes typisk opp mot et \textit{repository}, og brukes for å aksessere dette \textit{repositoryet} fra terminalen. Følg disse stegene for å lage et \textit{classic personal access token}:

\begin{enumerate}
    \item Trykk på profilbildet ditt (øvre høyre hjørne), og deretter \textit{Settings}.
    \item Deretter trykker du på \textit{Developer settings} i den venstre sidebaren.
    \item Deretter trykker du på \textit{Personal access tokens}, og så \textit{Tokens (classic)} i den venstre sidebaren.
    \item Trykk på \textit{Generate new token}, og deretter \textit{Generate new token (classic)}.
    \item Under \textit{Note}, skriv inn et navn til ditt \textit{token}. Dette kan f.eks. være det samme som navnet på \textit{repositoryet} du generelt kommer til å bruke det til.
    \item Under \textit{Expiration} velger du hvor lenge det skal være gyldig.
    \item Under \textit{Select scopes} velger du de rettighetene du vil at \textit{tokenet} skal ha. Til våre formål i TTK4235 kan det være greit å ha så mange rettigheter som mulig, men dersom sikkerhet hadde vært av større viktighet, hadde vi begrenset dette også. Kryss derfor av i alle boksene.
    \item Trykk på \textit{Generate token} for å fullføre prosessen. Husk å kopiere \textit{tokenet} et trygt sted, ettersom du ikke får muligheten til å se det igjen. Det er også alltid mulig å lage et nytt \textit{token} dersom uhellet skulle være ute.
\end{enumerate}

Nå er vi i stand til å klone \textit{repositoryet} vårt ved å bruke følgende kommando: \verb|git clone https://github.com/user_name/repository_name.git|, der "user\_name" og "repository\_name" byttes ut med henholdvis GitHub-brukernavnet og \textit{repository}-navnet ditt. Her vil du bli spurt om å skrive inn brukernavnet ditt og ditt passord (her skriver du inn ditt \textit{token}), og så har du klonet ditt \textit{repository}. 

Nå kan du for eksempel lage en enkel tekstfil i \textit{repositoryet} via terminalen, og deretter bruke \verb|git add file_name| for å legge til denne filen i \textit{staging area}. Så kan du \textit{comitte} gjennom kommandoen \verb|git commit -m "first commit"|, og deretter \textit{pushe} ved kommandoen \verb|git push|. Du har nå laget et GitHub-\textit{repository}, og gjort en \textit{commit}. Hvis du tar en titt på hovedsiden til ditt \textit{repository} vil du nå se endringene som har blitt gjort. Arbeidsflyten ellers er lik som forklart tidligere i denne øvingen, bortsett fra at \textit{merge} med fordel kan gjøres i nettsiden til selve \textit{repositoryet}.

I dette eksempelet gjorde vi en \textit{commit} rett inn i hovedgrenen \verb|main|. Dette bør man egentlig aldri gjøre, og vi anbefaler å gjøre \textit{commits} i andre grener, og deretter å \textit{merge} i GitHub sin nettleser. Dette er en svært vanlig konvensjon som hovedsaklig går ut på å holde \verb|main| kjørbar. I tillegg vil vi nevne at dere {\bf{ikke}} bør bruke \textit{global credentials} (f.eks. \verb|--global|-flagget i \verb|git| \verb|config|) når dere jobber på offentlige datamaskiner, noe dere vil gjøre i TTK4235. For mer informasjon om \verb|Git| og \verb|Git|-konvensjoner anbefaler vi å ta en titt på boken \href{https://git-scm.com/book/en/v2}{{\it Pro Git}} skrevet av Scott Chacon, som er tilgjengelig gratis på nett.

%\subsection*{Frivillig: \texttt{git rebase}}

% Som dere så, gjorde \verb|git merge| akkurat det vi ville - vi fikk slått sammen kodebasene, og samlet den oppdaterte koden inn i en ny \textit{commit}. Allikevel vil \verb|git| huske på splitten vi hadde. Altså, vil \verb|git| skrive dette i historieboka:

% \begin{itemize}
%     \item Var enige ved "classic code example".
%     \item Divergerte i to grener en stund.
%     \item Slo sammen grenene.
%     \item Lagde ny \textit{commit} "merged work".
% \end{itemize}

% \subsection*{Gjemt historie}

% Om dere gjorde den frivillige oppgaven, kall først \verb|git checkout master|, også \verb|git reset --hard e3192ad|, slik at historikken ser slik ut igjen:

% \texttt{* e3192ad (HEAD -> master) merged work}\newline
% \texttt{|\textbackslash}\newline
% \texttt{| * 57b9e8c (other) greet mars as well}\newline
% \texttt{* | af4c17f assert truth}\newline
% \texttt{|/}\newline
% \texttt{* f330827 classic example code}\newline
% \texttt{* 1c9576f added main.c}\newline

% Dette illustrer en veldig nyttig egenskap ved \verb|git|; ingen ferdige \textit{commits} slettes helt før de samles inn av \verb|git| sin søppelhåndterer - som tar 90 dager, så lenge standardverdien ikke endres. En \textit{commit} vil slettes når denne perioden er utløpt hvis det ikke er noen grener som inkluderer denne \textit{commiten}.

% Noe som er nyttig å vite, er at selv om \verb|git| ikke har slettet en commit, vil den ikke vises via \verb|git lg|. Dette er fordi \textit{commits} uten en gren blir regnet som "dangling", og \verb|git| vil i utgangspunktet ikke vise disse. For å alle \textit{dangling commits} kan man kalle \verb|git fsck --lost-found|, eller \verb|git reflog| for å se alt man har gjort i det siste.

% \subsection*{\texttt{git remote, git push, git fetch}}

% \subsection*{\texttt{.gitignore}}

